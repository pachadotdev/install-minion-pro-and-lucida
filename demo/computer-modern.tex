\documentclass[letterpaper]{article}

\usepackage{amssymb}
\usepackage{url}
\usepackage{hyperref}

% set text colour to #2c2e35
\usepackage{xcolor}
\definecolor{dark}{HTML}{2c2e35}
\color{dark}

% set margins to 1 inch
\usepackage[margin=1in]{geometry}

% set indent to 0
\setlength{\parindent}{0pt}

% set paragraph spacing to 1em
\setlength{\parskip}{1em}

\title{Imitating the Typography From Classic Hindawi Journals
(i.e., from `Fixed Points as Nash Equilibria' (Torres-Mart\'inez, 2006))}

\author{Pach\'a (\url{https://pacha.dev})}

\begin{document}

\maketitle

This uses Computer Modern (TeX default).

Let $Y \subset \mathbb{R}^n$ be a convex set. A function
$v: Y \rightarrow \mathbb{R}$ is \emph{quasiconcave} if, for each
$\lambda \in (0,1)$, we have
$v(\lambda y_1 + (1 - \lambda y_2)) \geq \min\{v(y_1),\: v(y_2) \}$, for all
$(y_1, y_2) \in Y \times Y$.

\emph{[Nash-2.]} Given $\mathcal{G}=\{I,S_i,V^i\}$, suppose each set
$S_i \in \mathcal{H}$ and that objective functions are continuous in its
domain and \emph{quasiconcave} in its own strategy. Then there is a Nash
equilibrium for $\mathcal{G}$.

\emph{[Kakutani.]} Given $X \in \mathcal{H}$, every closed-graph correspondence
$\Phi: X \twoheadrightarrow X$, with $\Phi(x) \in \mathcal{H}$ for all
$x \in X$, has a fixed point, provided that
$\Phi(x) = \prod_{j=1}^{m} \pi_j^m(\Phi(x))$ for each
$x \in X \subset \mathbb{R}^m$.

The article proves that \emph{[Nash-2]} $\rightarrow$ \emph{[Kakutani]}.

Ok, sufficient display of math symbols.

See \url{https://github.com/pachadotdev/varsityblues} for a set of complete
LaTeX templates to be used with R Markdown or Quarto.

\end{document}
